%*******************************************************
% Abstract in English
%*******************************************************
\pdfbookmark[0]{Abstract}{Abstract}


\begin{otherlanguage}{american}
	\chapter*{Abstract}
	Nowadays , enterprises are shifting more and more towards data-driven decision making. They are using this data to build applications that support them with business intelligence. This allows them to tackle business challenges more effectively , improve efficiency and increase profitability. However , the key problem in achieving these benefits lies in the abundance of data and the form it comes in. Transforming raw data into useful insights takes time and resources to implement. When companies develop these solutions , requirements often change over time. Depending on the architecture decisions used , these solutions may not be easily adaptable to change, forcing companies to invest even more resources to adjust them. This makes building and maintaining such systems costly and complex. Moreover,many enterprises also rely on AI models or machine learning algorithms to extract deeper insights from raw data and convert it into labeled data. In some cases , these models must be adjusted based on real-time data as well , in order to continuously improve their accuracy. What begins as a simple ETL process therefore evolves into a data classification problem with continuous learning on top. This solution consists of multiple modules that are closely coupled , meaning that a change in one module directly affects the others, adding further complexity.
	\smallskip

	This thesis will focus on addressing the previously described problem by designing a modular data classification pipeline with continuous learning. This research will be supported by a systematic literature review to identify the most relevant and modern solution available. The proposed architecture will then be validated through a real-world enterprise use case to demonstrate its effectiveness in practical conditions.
	\smallskip

	This data classification pipeline is a relatively recent approach that can bring significant results, by enabling frequent adjustments within the system through its modular architecture. The result will be a reliable data classification pipeline design that is adaptable to change and effectively addresses real-word challenges.

\end{otherlanguage}
